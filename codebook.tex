\documentclass[landscape, 8pt,twocolumn,oneside, a4paper]{article}
\title {NCTU\_Jaguar Codebook}
\usepackage{parskip}
\usepackage{xeCJK} 
\setCJKmainfont{Noto Sans CJK TC}
\setmonofont{Courier New}
\usepackage {listings}
\usepackage {color}
\usepackage[x11names]{xcolor}
\usepackage [left=1.0cm, right=1.0cm, top=2.0cm, bottom=0.5cm]{geometry}
\definecolor {mygreen}{rgb}{0,0.6,0}
\definecolor {mygray}{rgb}{0.5,0.5,0.5}
\definecolor {mymauve}{rgb}{0.58,0,0.82}
\usepackage{fancyheadings}
\rhead{\thepage}
\chead{init? array size? \texttt{x, y}? overflow\texttt{int}? 1-based? OK, Submit!}
\lhead{NCTU\_Jaguar}
\pagestyle{fancy}
\cfoot{}
\setlength{\headsep}{5pt}
\setlength{\textheight}{540pt}

\lstset {
    language=C++,                 % the language of the code
    basicstyle=\footnotesize\ttfamily,          % the size of the fonts that are used for the code
    numberstyle=\footnotesize,          % the size of the fonts that are used for the line-numbers
    stepnumber=1,                 % the step between two line-numbers. If it's 1, each line  will be numbered
    numbersep=5pt,                  % how far the line-numbers are from the code
    backgroundcolor=\color{white},        % choose the background color. You must add \usepackage{color}
    showspaces=false,               % show spaces adding particular underscores
    showstringspaces=false,           % underline spaces within strings
    showtabs=false,               % show tabs within strings adding particular underscores
    frame=false,                    % adds a frame around the code
    tabsize=2,                    % sets default tabsize to 2 spaces
    captionpos=b,                 % sets the caption-position to bottom
    breaklines=true,                % sets automatic line breaking
    breakatwhitespace=false,            % sets if automatic breaks should only happen at whitespace
    escapeinside={\%*}{*)},           % if you want to add a comment within your code
    morekeywords={*},               % if you want to add more keywords to the set
    keywordstyle=\bfseries\color{Blue1},
    commentstyle=\itshape\color{Red4},
    stringstyle=\itshape\color{Green4},
}

\begin {document}
\thispagestyle{fancy}
{ \Huge NCTU\_Jaguar}
\tableofcontents

\section{DataStructure}
\subsection{Ext Heap}
\lstinputlisting [language=c++] {"code/DataStructure/ext_heap.cpp"}
\subsection{KDTree}
\lstinputlisting [language=c++] {"code/DataStructure/KDTree.cpp"}
\subsection{Link Cut Tree}
\lstinputlisting [language=c++] {"code/DataStructure/link_cut_tree.cpp"}
\subsection{SparseTable}
\lstinputlisting [language=c++] {"code/DataStructure/SparseTable.h"}
\subsection{Treap}
\lstinputlisting [language=c++] {"code/DataStructure/Treap.cpp"}
\section{Flow}
\subsection{Dinic}
\lstinputlisting [language=c++] {"code/Flow/Dinic.cpp"}
\subsection{Gomory Hu}
\lstinputlisting [language=c++] {"code/Flow/Gomory_Hu.cpp"}
\subsection{Min Cost Flow}
\lstinputlisting [language=c++] {"code/Flow/Min Cost Flow.cpp"}
\subsection{SW-mincut}
\lstinputlisting [language=c++] {"code/Flow/SW-mincut.cpp"}
\section{Geometry}
\subsection{2Dpoint}
\lstinputlisting [language=c++] {"code/Geometry/2Dpoint.cpp"}
\subsection{Circumcentre}
\lstinputlisting [language=c++] {"code/Geometry/circumcentre.cpp"}
\subsection{ConvexHull}
\lstinputlisting [language=c++] {"code/Geometry/ConvexHull.cpp"}
\subsection{Half Plane Intersection}
\lstinputlisting [language=c++] {"code/Geometry/half_plane_intersection.cpp"}
\subsection{Intersection Of Two Circle}
\lstinputlisting [language=c++] {"code/Geometry/Intersection_of_two_circle.cpp"}
\subsection{Intersection Of Two Lines}
\lstinputlisting [language=c++] {"code/Geometry/Intersection_of_two_lines.cpp"}
\subsection{Smallest Circle}
\lstinputlisting [language=c++] {"code/Geometry/Smallest_Circle.cpp"}
\section{Graph}
\subsection{BCC Edge}
\lstinputlisting [language=c++] {"code/Graph/BCC_edge.cpp"}
\subsection{Dijkstra}
\lstinputlisting [language=python] {"code/Graph/Dijkstra.py"}
\subsection{LCA}
\lstinputlisting [language=c++] {"code/Graph/LCA.cpp"}
\subsection{MaximumClique}
\lstinputlisting [language=c++] {"code/Graph/MaximumClique.cpp"}
\subsection{MinimumSteinerTree}
\lstinputlisting [language=c++] {"code/Graph/MinimumSteinerTree.cpp"}
\subsection{Min Mean Cycle}
\lstinputlisting [language=c++] {"code/Graph/Min_mean_cycle.cpp"}
\subsection{Tarjan}
\lstinputlisting [language=c++] {"code/Graph/Tarjan.cpp"}
\subsection{TwoSAT}
\lstinputlisting [language=c++] {"code/Graph/TwoSAT.cpp"}
\section{Matching}
\subsection{KM}
\lstinputlisting [language=c++] {"code/Matching/KM.cpp"}
\subsection{Maximum General Matching}
\lstinputlisting [language=c++] {"code/Matching/Maximum_General_Matching.cpp"}
\subsection{Minimum General Weighted Matching}
\lstinputlisting [language=c++] {"code/Matching/Minimum_General_Weighted_Matching.cpp"}
\subsection{Stable Marriage}
\lstinputlisting [language=c++] {"code/Matching/Stable Marriage.cpp"}
\section{Math}
\subsection{Ax+by=gcd}
\lstinputlisting [language=c++] {"code/Math/ax+by=gcd.cpp"}
\subsection{FFT}
\lstinputlisting [language=c++] {"code/Math/FFT.cpp"}
\subsection{FWHT}
\lstinputlisting [language=c++] {"code/Math/FWHT.cpp"}
\subsection{GaussElimination}
\lstinputlisting [language=c++] {"code/Math/GaussElimination.cpp"}
\subsection{Inverse}
\lstinputlisting [language=c++] {"code/Math/inverse.cpp"}
\subsection{Karatsuba}
\lstinputlisting [language=c++] {"code/Math/Karatsuba.cpp"}
\subsection{LinearPrime}
\lstinputlisting [language=c++] {"code/Math/LinearPrime.cpp"}
\subsection{Miller-Rabin}
\lstinputlisting [language=c++] {"code/Math/Miller-Rabin.cpp"}
\subsection{Mobius}
\lstinputlisting [language=c++] {"code/Math/Mobius.cpp"}
\subsection{PollardRho}
\lstinputlisting [language=c++] {"code/Math/pollardRho.cpp"}
\subsection{Sprague-Grundy}
\lstinputlisting [language=c++] {"code/Math/Sprague-Grundy.cpp"}
\subsection{Theorem}
\lstinputlisting [language=c++] {"code/Math/theorem.cpp"}
\section{Other}
\subsection{Count Spanning Tree}
\lstinputlisting [language=c++] {"code/Other/count_spanning_tree.cpp"}
\subsection{CYK}
\lstinputlisting [language=c++] {"code/Other/CYK.cpp"}
\subsection{DigitCounting}
\lstinputlisting [language=c++] {"code/Other/DigitCounting.cpp"}
\subsection{DP-optimization}
\lstinputlisting [language=c++] {"code/Other/DP-optimization.txt"}
\subsection{Dp1D1D}
\lstinputlisting [language=c++] {"code/Other/Dp1D1D.cpp"}
\subsection{ManhattanMST}
\lstinputlisting [language=c++] {"code/Other/ManhattanMST.cpp"}
\section{String}
\subsection{AC}
\lstinputlisting [language=c++] {"code/String/AC.cpp"}
\subsection{BWT}
\lstinputlisting [language=c++] {"code/String/BWT.cpp"}
\subsection{KMP}
\lstinputlisting [language=c++] {"code/String/KMP.h"}
\subsection{PalindromicTree}
\lstinputlisting [language=c++] {"code/String/PalindromicTree.cpp"}
\subsection{SAM}
\lstinputlisting [language=c++] {"code/String/SAM.cpp"}
\subsection{Smallest Rotation}
\lstinputlisting [language=c++] {"code/String/smallest_rotation.cpp"}
\subsection{Suffix Array}
\lstinputlisting [language=c++] {"code/String/suffix_array.cpp"}
\subsection{Z-value}
\lstinputlisting [language=c++] {"code/String/Z-value.cpp"}

\end{document}

